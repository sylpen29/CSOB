\documentclass[french]{article}
\usepackage[utf8]{inputenc}
\usepackage[T1]{fontenc}

\usepackage{natbib}
\usepackage[vmargin=3cm,left=4cm,right=4cm]{geometry}

\usepackage{babel}

\usepackage[table,xcdraw]{xcolor}

\usepackage{graphicx}
\usepackage{caption} 
\captionsetup{justification=centering}
\usepackage{subcaption}
% \usepackage{hyperref}
\usepackage[hidelinks]{hyperref}

\begin{document}

%###############################################
\begin{titlepage}

\newcommand{\HRule}{\rule{\linewidth}{0.5mm}} % Defines a new command for the horizontal lines, change thickness here

\center % Center everything on the page
 
%----------------------------------------------------------------------------------------
%	Section Titre
%----------------------------------------------------------------------------------------
\HRule \\[0.4cm]
\vspace{1cm}
{ \huge \bfseries Breizhibus : Cahier des charges}\\ % Title of your document
\vspace{1cm}
\HRule \\[1cm]
 
%----------------------------------------------------------------------------------------
%	Section auteur
%----------------------------------------------------------------------------------------
\vspace{1cm}

\Large \today

\vspace{3cm}

\begin{minipage}{0.4\textwidth}
\begin{center}
\Large \textbf{Auteurs :}\\
\vspace{0.5cm}
Franky \textsc{Tanguy} \\
Hervé \textsc{Poirier}\\
Fabio \textsc{Cassiano}
\end{center}
\end{minipage}

\vspace{5cm}

\begin{figure}[!ht]
    %\hspace*{-0.5cm}
	\includegraphics[height=0.1\columnwidth]{Image/logo/logo_simplon.png}
	\hspace*{0.5cm}
	\includegraphics[height=0.12\columnwidth]{Image/logo/logo_Isen.png}
	\hspace*{0.5cm}
	\includegraphics[height=0.1\columnwidth]{Image/logo/logo_microsoft.jpg}
\end{figure}

\vfill

\end{titlepage}

\newpage

\tableofcontents

\newpage

\section{Introduction}

\textbf{Nom du projet :} Breizhibus Online

Breizhibus Online est une application Web, permettant aux usager de consulter les lignes de bus et les arrêts que propose la société. L'application doit également permettre aux employés agréés de mettre à jour les données par le biais d'une connexion dédiée.

\subsection{Motivations}

Le motivations qui ont mené à ce projet sont les suivantes :

\begin{itemize}
    \item Faciliter l'accès aux informations pour les usagers.
    \item Faciliter la mise à jour des données concernant les lignes et les bus qui y sont associés.
\end{itemize}

\subsubsection{Le client}

Breizhibus est une société proposant un service de transport public de voyageurs. La société est composée de plusieurs véhicules de type bus.

\subsubsection{Le problème}

Le client souhaite informatiser la gestion de ses lignes de bus. La société souhaite proposer une application qui :
\begin{itemize}
    \item Permette aux usagers de consulter les lignes de bus et les arrêts.
    \item Permette à ses employés de mettre à jour simplement les données.
\end{itemize}

\subsubsection{L'existant}

Un prestataire précédent a informatisé toutes les données sous la forme d'une base MySQL. Mais la mise à jour ne se fait qu'en requête SQL. Cependant personne n'ayant cette compétence en interne, cela oblige le client à passer par le prestataire, qui facture chaque intervention.

\subsubsection{Le besoin non satisfait}

Il doit être possible pour les employés de Breizhibus de mettre à jour les données de la base sans avoir recours à des requêtes SQL, de manière simple et intuitive.

\subsubsection{Les objectifs}

La mission est de développer l'application web à partir de la base Breizhibus existante.

\subsection{Précisions sur le client}

Le client a précisé certaines contraintes, au niveau de l'interface professionnel : 
\begin{itemize}
    \item Un espace de connexion pour les employés agréés, permettant d'ajouter/éditer les bus avec assignation de lignes.
    \item Un bus ne peut avoir qu'une seule ligne
    \item Une ligne peut avoir plusieurs bus qui lui sont assignés
\end{itemize}

Côté utilisateur :
\begin{itemize}
    \item Consulter les lignes de bus et les arrêts qu'elles décernent
    \item Application intuitive, simple, et épurée.
\end{itemize}

\subsubsection{Marché}

Le fait de faciliter l'accès aux informations permettra d'avoir plus de visibilité et d'accroître le nombre d'utilisateur.

\section{Documentation}

\subsection{Terminologie métier}

La société propose des véhicules de type bus dont les trajets sont considérés comme des lignes, et on retrouve des arrêts le long de ces lignes.

\subsection{Profil des utilisateurs finaux}

Deux profils d’utilisateurs ont été identifiés :
\begin{itemize}
    \item Utilisateurs (grand public) : seuls compétences nécessaires, savoir naviguer sur un site internet.
    \item Professionnel :  avoir un compte dans la société afin de pouvoir s’identifier,  et être habilité à apporter des modifications sur le réseau de bus.
\end{itemize}

\section{Fonctions à réaliser}

\subsection{Ce que le système doit faire}

\begin{table}[!htbp]
\centering
\begin{tabular}{|l|l|p{0.28\linewidth}|p{0.28\linewidth}|}
\hline
\rowcolor[HTML]{D7D7D7} 
\textbf{Priorité} & \textbf{Fonction}     & \textbf{Description}                                                                                                         & \textbf{Critère de performances}                                                                                                                                                                  \\ \hline
1                 & Modification de base  & Il doit être possible de modifier, supprimer ou actualiser les bus dans la base de données sans avoir de compétences en SQL. & Une modification/actualisation effective en moins de 1 minute.                                                                                                                                    \\ \hline
2                 & IHM Professionnel     & Interface graphique (tableau de bord) permettant la manipulation de la base de données.                                      & Un affichage des lignes, des bus, et des arrêts dans un tableau. Un bouton permettant la modification des bus, et un bouton permettant la saisie d’un nouveau bus avec l’attribution d’une ligne. \\ \hline
3                 & Page d’identification & Une page permettant aux agents agréés de s’identifier pour accéder au tableau de bord.                                       & Des entrées textuels pour le nom d’utilisateur, et le mot de passe. Un bouton permettant la validation.                                                                                           \\ \hline
4                 & IHM Utilisateur       & Interface graphique pour tous les utilisateurs permettant d’accéder aux informations des bus, des lignes, et des arrêts.     & Une visualisation du réseau de bus général. Une sélection des informations par ligne, bus ou arrêt. Affichage des informations sous forme de tableau.                                             \\ \hline
\end{tabular}
\caption{Fonctionnalité de l'application et critère de performances associé.}
\end{table}

\subsection{Ce que le système ne doit pas faire}

Les employés n’auront pas accès à la création de compte pouvant modifier la base de données. 
Ils ne pourront pas modifier l’architecture de la base de données:  créer/supprimer des tables.

\section{Contraintes du système}

\subsection{Contraintes matérielles}

L'application pourra uniquement être utilisé sur les navigateurs internet suivants :
\begin{itemize}
    \item Firefox v. 97.0.0 (64 bits)
    \item Google Chrome v. 98.0.0 (64 bits)
\end{itemize}

\subsection{Contraintes logicielles}

L'application doit être compatible avec la base de données MySQL, qui est le système de base de données utilisé par le client.

\subsection{Contraintes fonctionnelles}

Les utilisateurs ne doivent pas avoir accès à l'interface de gestion destinées aux employés.

\subsection{Contraintes d'ergonomie}

La partie utilisateur doit être simple, intuitive et accessible depuis chez eux. L'interface professionnel doit être sécurisée par le biais d'un connexion, elle doit être rapide et compréhensible pour le métier.

\subsection{Contraintes techniques}

Le produit proposé sera développé en python 3, en utilisant un serveur flask. L’interface du produit sera proposée en html5 et css3, et pourra communiquer avec la base de données MySQL par le programme python.

\subsection{Autres facteurs de qualité exigés par le client}

L'accès à la base de données doit être simplifié en écriture avec des commandes restreintes ne permettant pas de faire de mauvaises manipulations de la part des employés.

\end{document}
